% Options for packages loaded elsewhere
\PassOptionsToPackage{unicode}{hyperref}
\PassOptionsToPackage{hyphens}{url}
%
\documentclass[
]{article}
\usepackage{amsmath,amssymb}
\usepackage{iftex}
\ifPDFTeX
  \usepackage[T1]{fontenc}
  \usepackage[utf8]{inputenc}
  \usepackage{textcomp} % provide euro and other symbols
\else % if luatex or xetex
  \usepackage{unicode-math} % this also loads fontspec
  \defaultfontfeatures{Scale=MatchLowercase}
  \defaultfontfeatures[\rmfamily]{Ligatures=TeX,Scale=1}
\fi
\usepackage{lmodern}
\ifPDFTeX\else
  % xetex/luatex font selection
\fi
% Use upquote if available, for straight quotes in verbatim environments
\IfFileExists{upquote.sty}{\usepackage{upquote}}{}
\IfFileExists{microtype.sty}{% use microtype if available
  \usepackage[]{microtype}
  \UseMicrotypeSet[protrusion]{basicmath} % disable protrusion for tt fonts
}{}
\makeatletter
\@ifundefined{KOMAClassName}{% if non-KOMA class
  \IfFileExists{parskip.sty}{%
    \usepackage{parskip}
  }{% else
    \setlength{\parindent}{0pt}
    \setlength{\parskip}{6pt plus 2pt minus 1pt}}
}{% if KOMA class
  \KOMAoptions{parskip=half}}
\makeatother
\usepackage{xcolor}
\usepackage[margin=1in]{geometry}
\usepackage{graphicx}
\makeatletter
\def\maxwidth{\ifdim\Gin@nat@width>\linewidth\linewidth\else\Gin@nat@width\fi}
\def\maxheight{\ifdim\Gin@nat@height>\textheight\textheight\else\Gin@nat@height\fi}
\makeatother
% Scale images if necessary, so that they will not overflow the page
% margins by default, and it is still possible to overwrite the defaults
% using explicit options in \includegraphics[width, height, ...]{}
\setkeys{Gin}{width=\maxwidth,height=\maxheight,keepaspectratio}
% Set default figure placement to htbp
\makeatletter
\def\fps@figure{htbp}
\makeatother
\setlength{\emergencystretch}{3em} % prevent overfull lines
\providecommand{\tightlist}{%
  \setlength{\itemsep}{0pt}\setlength{\parskip}{0pt}}
\setcounter{secnumdepth}{-\maxdimen} % remove section numbering
\ifLuaTeX
  \usepackage{selnolig}  % disable illegal ligatures
\fi
\usepackage{bookmark}
\IfFileExists{xurl.sty}{\usepackage{xurl}}{} % add URL line breaks if available
\urlstyle{same}
\hypersetup{
  pdftitle={Trabalho final sobre a pandemia da COVID 19},
  hidelinks,
  pdfcreator={LaTeX via pandoc}}

\title{Trabalho final sobre a pandemia da COVID 19}
\author{}
\date{\vspace{-2.5em}22 de junho de 2025}

\begin{document}
\maketitle

\section{Relatório}\label{relatuxf3rio}

Disciplina de Estatística computacional Grupo formado por: Bernardo
Monteiro Rocha - 202410221-26 Diogo Barbosa Silva Sousa - 202210091-32

\section{1. Introdução}\label{introduuxe7uxe3o}

Este trabalho tem por objetivo analisar a pandemia da COVID 19 no
intervalo entre 27/03/2020 até 25/06/2020 e comparar as regiões para ver
a periculosidade em cada estado.

\section{2. Materiais e Métodos}\label{materiais-e-muxe9todos}

\subsection{2.1. População de Estudo}\label{populauxe7uxe3o-de-estudo}

A população é um dataset que abrange apenas o Brasil no intervalo de
tempo citado na introdução. A fonte é o site openDataSUS.

\subsection{2.2. Análise}\label{anuxe1lise}

Foram analisados os casos acumulados, óbitos e a proporção com a
população para verificar o grau de periculosidade nos estados.

\subsection{2.3. Analise estatística}\label{analise-estatuxedstica}

Foram criadas as variáveis obitosAcumuladoPelaPopulacao e
casosAcumuladosPelaPopulacao para criar as taxas de óbitos e casos pela
população.

\subsection{2.4. Histogramas}\label{histogramas}

\includegraphics{relatorio_COVID_19_files/figure-latex/unnamed-chunk-3-1.pdf}
\includegraphics{relatorio_COVID_19_files/figure-latex/unnamed-chunk-3-2.pdf}

\subsection{2.5 Resumindo dados}\label{resumindo-dados}

Resumo na semana 26 da pandemia

\begin{verbatim}
## # A tibble: 5 x 4
##   regiao       Populacao_Total Casos_Acumulados Obitos_Acumulados
##   <chr>                  <dbl>            <dbl>             <dbl>
## 1 Centro-Oeste        15991427            77044              1407
## 2 Nordeste            56534807           417343             17550
## 3 Norte               18355938           240655              9128
## 4 Sudeste             87179992           426844             25476
## 5 Sul                 29209938            62267              1315
\end{verbatim}

\end{document}
